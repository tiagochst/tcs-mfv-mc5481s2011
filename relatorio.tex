\documentclass[10pt,a4paper]{article}
\usepackage[T1]{fontenc}
\usepackage[brazil]{babel}
\usepackage[utf8]{inputenc}


\usepackage{ae,aecompl}
\usepackage{pslatex}
\usepackage{epsfig}
\usepackage{geometry}
\usepackage{url}
\usepackage{textcomp}
\usepackage{ae}
\usepackage{subfig}
\usepackage{indentfirst}
\usepackage{textcomp}
\usepackage{color}
\usepackage{setspace}
\usepackage{verbatim}
\usepackage{mathtools}
\usepackage{amsmath}


\usepackage[compact]{titlesec}
\titlespacing{\section}{0pt}{*0}{*0}
\titlespacing{\subsection}{0pt}{*0}{*0}
\titlespacing{\subsubsection}{0pt}{*0}{*0}

\linespread{1.5}
\geometry{ 
  a4paper,	% Formato do papel
  tmargin=25mm,	% Margem superior
  bmargin=25mm,	% Margem inferior
  lmargin=20mm,	% Margem esquerda
  rmargin=20mm,	% Margem direita
  footskip=10mm	% Espaço entre o rodapé e o fim do texto
}
%  ABACO -- Conjunto de macros para desenhar o 'abaco

%  Desenho original de Hans Liesenberg

%  Macros de Tomasz Kowaltowski

%  DCC -- IMECC -- UNICAMP

%  Mar,co de 1988  --  Vers~ao 1.0

% Ajustado para LaTeX da SUN -- Mar,co de 1991

% ---------------------------------------------------------

%  Chamada:   \ABACO{d1}{d2}{d3}{d4}{esc}
%             com:  di's -- os quatro d'igitos;
%	           esc  -- fator de escala

% ---------------------------------------------------------

%  DEFINI,C~OES AUXILIARES

% ---------------------------------------------------------


%  Forma o d'igito pequeno (0 ou 1)

\newcommand{\ABACODP}[1]{%
%
\thicklines
%    
\begin{picture}(8,0)
    \ifcase#1{   %  caso 0
       \put(0,0)    {\line(1,0){4}}
       \multiput(5,0)(2,0){2}{\oval(2,4)}}
    \or{         %  caso 1
       \put(2,0)    {\line(1,0){4}}
       \multiput(1,0)(6,0){2}{\oval(2,4)}}
    \fi
\end{picture}
    } % \ABACODP

% Forma o d'igito grande (0 a 4)

\newcommand{\ABACODG}[1]{%
%
\thicklines
%    
\begin{picture}(14,0)
    \ifcase#1{   % caso 0
       \multiput(1,0)(2,0){5}{\oval(2,4)}}
       \put(10,0)   {\line(1,0){4}}
    \or{         % caso 1
       \multiput(1,0)(2,0){4}{\oval(2,4)}}
       \put(8,0)   {\line(1,0){4}}
       \put(13,0)   {\oval(2,4)}
    \or{         % caso 2
       \multiput(1,0)(2,0){3}{\oval(2,4)}
       \put(6,0)   {\line(1,0){4}}
       \multiput(11,0)(2,0){2}{\oval(2,4)}}
    \or{         % caso 3
       \multiput(1,0)(2,0){2}{\oval(2,4)}
       \put(4,0)   {\line(1,0){4}}
       \multiput(9,0)(2,0){3}{\oval(2,4)}}
    \or{         % caso 4
       \put(1,0)  {\oval(2,4)}}
       \put(2,0)   {\line(1,0){4}}
       \multiput(7,0)(2,0){4}{\oval(2,4)}
    \fi
\end{picture}
    } % \ABACODG
       
% Forma um d'igito (0 a 9)

\newcommand{\ABACOD}[1]{%
%
    \ifnum#1>9
       \errmessage{#1: Argumento invalido para ABACO}
    \fi
    \ifnum#1<0
       \errmessage{#1: Argumento invalido para ABACO}
    \fi
%
\begin{picture}(24,0)
%    
    \ifnum#1<5
       \put(16,0) {\ABACODP{0}}
    \else   
       \put(16,0) {\ABACODP{1}}
    \fi
%    
    \ifnum#1<5
       \put(0,0)  {\ABACODG{#1}}
    \else
       \ifcase#1\or \or \or \or
          \or  \put(0,0)  {\ABACODG{0}}
          \or  \put(0,0)  {\ABACODG{1}}
          \or  \put(0,0)  {\ABACODG{2}}
          \or  \put(0,0)  {\ABACODG{3}}
          \or  \put(0,0)  {\ABACODG{4}}
       \fi
    \fi   
\end{picture}
    } % \ABACOD
    
% -------------------------------------------------

%  DEFINI,C~AO PRINCIPAL
    
\newcommand{\ABACO}[5]{%
    \setlength{\unitlength}{#5mm}
%
    \thinlines
%   
\begin{picture}(28,25)
%   
% moldura
%
% externa
%
        \put(0,0)            {\line(0,1){25}}
        \put(0,0)            {\line(1,0){28}}
        \put(28,0)           {\line(0,1){25}}
        \put(0,25)           {\line(1,0){28}}
% interna
        \put(2,2)            {\line(0,1){21}}
	\put(26,2)           {\line(0,1){21}}
	\put(16,2)           {\line(0,1){21}}
	\put(18,2)           {\line(0,1){21}}
	\put(2,2)            {\line(1,0){14}}
	\put(16,2)           {\line(1,-1){1}}
	\put(17,1)           {\line(1,1){1}}
	\put(18,2)           {\line(1,0){8}}
	\put(2,23)           {\line(1,0){14}}
	\put(16,23)          {\line(1,1){1}}
	\put(17,24)          {\line(1,-1){1}}
	\put(18,23)          {\line(1,0){8}}
	\put(0,0)            {\line(1,1){2}}
	\put(0,25)           {\line(1,-1){2}}
	\put(28,0)           {\line(-1,1){2}}
	\put(28,25)          {\line(-1,-1){2}}
%
%   
% d'igitos
%
%   
       \put(2,20)  {\ABACOD{#1}}
       \put(2,15)  {\ABACOD{#2}}
       \put(2,10)  {\ABACOD{#3}}
       \put(2,5)   {\ABACOD{#4}}
%      
\end{picture}
    } % \ABACO
    
 
\renewcommand{\thetable}{\Roman{table}}
\newcommand{\x} {$\bullet$}


\begin{document}
% CAPA
  \thispagestyle{empty}
  
  \begin{minipage}[h]{0.10\linewidth}
    \ABACO{1}{9}{6}{9}{0.5} 
  \end{minipage}
  \begin{minipage}[h!]{0.7\linewidth}
    \vspace*{\fill}
    \centering
    {\large \textbf{UNIVERSIDADE~ESTADUAL~DE~CAMPINAS}}\\ 
    {\large INSTITUTO~DE~COMPUTAÇÃO}                   
    \vspace*{\fill} 
  \end{minipage}
    \\\vspace{0.5cm}
  
  \begin{center} 
    \rule{11.0cm}{0.4pt}\vspace*{-\baselineskip}\vspace{-2.0pt}
    \rule{11.0cm}{1.6pt} \vspace*{-2.0pt}\\
      {\Large \textsc{Relatório do projeto de MC548}}\\
	\vspace*{-\baselineskip}\vspace{4.2pt} \rule{11.0cm}{0.4pt}
	\vspace*{-\baselineskip}\vspace{3.2pt} \rule{11.0cm}{1.6pt}\\
    {\textsl{}}
    \\\vspace{1cm}
    \begin{tabular}{ll}
   \textbf{Aluno}:   Murilo~Fossa~Vicentini &
      \textbf{RA}:          082335 \\ 
      \textbf{Aluno}:        Tiago~Chedraoui~Silva & 
      \textbf{RA}:        082941 \\
    \end{tabular}
  \end{center}
  \vspace{0.5cm}

  % Sumário
  \tableofcontents

\vspace{2mm}
\newpage

\section{Integrantes}
   \begin{tabular}{rl}
      \textbf{Aluno}:   Murilo~Fossa~Vicentini &
      \textbf{RA}:          082335 \\ 
      \textbf{Aluno}:        Tiago~Chedraoui~Silva & 
      \textbf{RA}:        082941 \\
    \end{tabular}


% ******************************************************
% Parte 1
% ******************************************************
 
\section{Parte 1}

\subsection{{[}nd30{]}}
\subsubsection*{Variáveis usadas no modelo}
\subsubsection*{Retrições do modelo}
\subsubsection*{Função objetivo}

\subsection{{[}mn27{]}}

\subsubsection*{Variáveis usadas no modelo}
\begin{itemize}
\item Para cada vértice $u \in V$ e para cada cor $k \in \{1 ... m\}$, criou-se
a variável binária $x_{uk}$. $x_{uk}=1$ se e somente se o vertíce $u$ foi
colorido com a cor $k$.

\item Criou-se uma variável binária $y_k$ para toda cor $k \in \{1 ... m\}$. $k=1$ se
e somente se pelo menos um vértice recebeu essa cor.
\end{itemize}

\subsubsection*{Retrições do modelo}
\begin{itemize}

\item Todo vértice deve receber exatamente uma cor
\begin{equation*}
  \sum_{k=1}^{m}x_{uk}=1, \forall u \in V
\end{equation*}

\item Se um vértice recebe a cor k, esta deve ser usada
\begin{equation*}
  x_{uk} \leq y_k, \forall u \in V, k \in \{1 ... m\}
\end{equation*}

\item Os Vértices vizinhos não podem ter a mesma cor
\begin{equation*}
  x_{uk} + x_{vk} \leq 1, \forall (u,v) \in E, k \in \{1 ... m\}
\end{equation*}
 
\end{itemize}


\subsubsection*{Função objetivo}
Objetivo: minimizar o número de cores usadas:
\begin{equation}
min\sum_{k=1}^{m}y_k
\end{equation}


\subsection{{[}ss2{]}}
\subsubsection*{Variáveis usadas no modelo}
\subsubsection*{Retrições do modelo}
\subsubsection*{Função objetivo}

\subsection{{[}ss15{]}}
\subsubsection*{Variáveis usadas no modelo}
\subsubsection*{Retrições do modelo}
\subsubsection*{Função objetivo}

\subsection{{[}mn22{]}}
\subsubsection*{Variáveis usadas no modelo}
\subsubsection*{Retrições do modelo}
\subsubsection*{Função objetivo}


% ******************************************************
% Parte 1 resultados
% ******************************************************

\subsection{Resultados}

% Export from lyx and paste here!

\begin{table}[h!]
\begin{centering}
\begin{tabular}{|c|c|c|c|}
\hline 
ID exercício & 1 & 2 & 3\tabularnewline
\hline 
\hline 
{[}nd30{]}  &  &  & \tabularnewline
\hline 
 {[}mn27{]} &  &  & \tabularnewline
\hline 
 {[}ss2{]} &  &  & \tabularnewline
\hline 
{[}ss15{]} &  &  & \tabularnewline
\hline 
 {[}mn22{]} &  &  & \tabularnewline
\hline 
\end{tabular}
\par\end{centering}

\caption{Resultados da parte 1}
\end{table}

% ******************************************************
% Parte 2
% ******************************************************

\subsection{Parte 2}



% ******************************************************
% REFERENCIAS BIBLIOGRÁFICAS
% ******************************************************
% \section{Referências}
\bibliographystyle{plain}
\begin{small}
  \bibliography{referencias}
\end{small}

\end{document}
