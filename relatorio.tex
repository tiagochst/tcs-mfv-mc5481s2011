\documentclass[10pt,a4paper]{article}
\usepackage[T1]{fontenc}
\usepackage[brazil]{babel}
\usepackage[utf8]{inputenc}


\usepackage{ae,aecompl}
\usepackage{pslatex}
\usepackage{epsfig}
\usepackage{geometry}
\usepackage{url}
\usepackage{textcomp}
\usepackage{ae}
\usepackage{subfig}
\usepackage{indentfirst}
\usepackage{textcomp}
\usepackage{color}
\usepackage{setspace}
\usepackage{verbatim}
\usepackage{mathtools}
\usepackage{amsmath}


\usepackage[compact]{titlesec}
\titlespacing{\section}{0pt}{*0}{*0}
\titlespacing{\subsection}{0pt}{*0}{*0}
\titlespacing{\subsubsection}{0pt}{*0}{*0}

\linespread{1.5}
\geometry{ 
  a4paper,	% Formato do papel
  tmargin=25mm,	% Margem superior
  bmargin=25mm,	% Margem inferior
  lmargin=20mm,	% Margem esquerda
  rmargin=20mm,	% Margem direita
  footskip=10mm	% Espaço entre o rodapé e o fim do texto
}
\include{abaco} 
\renewcommand{\thetable}{\Roman{table}}
\newcommand{\x} {$\bullet$}


\begin{document}
% CAPA
  \thispagestyle{empty}
  
  \begin{minipage}[h]{0.10\linewidth}
    \ABACO{1}{9}{6}{9}{0.5} 
  \end{minipage}
  \begin{minipage}[h!]{0.7\linewidth}
    \vspace*{\fill}
    \centering
    {\large \textbf{UNIVERSIDADE~ESTADUAL~DE~CAMPINAS}}\\ 
    {\large INSTITUTO~DE~COMPUTAÇÃO}                   
    \vspace*{\fill} 
  \end{minipage}
    \\\vspace{0.5cm}
  
  \begin{center} 
    \rule{11.0cm}{0.4pt}\vspace*{-\baselineskip}\vspace{-2.0pt}
    \rule{11.0cm}{1.6pt} \vspace*{-2.0pt}\\
      {\Large \textsc{Relatório do projeto de MC548}}\\
	\vspace*{-\baselineskip}\vspace{4.2pt} \rule{11.0cm}{0.4pt}
	\vspace*{-\baselineskip}\vspace{3.2pt} \rule{11.0cm}{1.6pt}\\
    {\textsl{}}
    \\\vspace{1cm}
    \begin{tabular}{ll}
   \textbf{Aluno}:   Murilo~Fossa~Vicentini &
      \textbf{RA}:          082335 \\ 
      \textbf{Aluno}:        Tiago~Chedraoui~Silva & 
      \textbf{RA}:        082941 \\
    \end{tabular}
  \end{center}
  \vspace{0.5cm}

  % Sumário
  \tableofcontents

\vspace{2mm}
\newpage

\section{Integrantes}
   \begin{tabular}{rl}
      \textbf{Aluno}:   Murilo~Fossa~Vicentini &
      \textbf{RA}:          082335 \\ 
      \textbf{Aluno}:        Tiago~Chedraoui~Silva & 
      \textbf{RA}:        082941 \\
    \end{tabular}


% ******************************************************
% Parte 1
% ******************************************************
 
\section{Parte 1}

\subsection{{[}nd30{]}}
\subsubsection*{Variáveis usadas no modelo}
\subsubsection*{Retrições do modelo}
\subsubsection*{Função objetivo}

\subsection{{[}mn27{]}}

\subsubsection*{Variáveis usadas no modelo}
\begin{itemize}
\item Para cada vértice $u \in V$ e para cada cor $k \in \{1 ... m\}$, criou-se
a variável binária $x_{uk}$. $x_{uk}=1$ se e somente se o vertíce $u$ foi
colorido com a cor $k$.

\item Criou-se uma variável binária $y_k$ para toda cor $k \in \{1 ... m\}$. $k=1$ se
e somente se pelo menos um vértice recebeu essa cor.
\end{itemize}

\subsubsection*{Retrições do modelo}
\begin{itemize}

\item Todo vértice deve receber exatamente uma cor
\begin{equation*}
  \sum_{k=1}^{m}x_{uk}=1, \forall u \in V
\end{equation*}

\item Se um vértice recebe a cor k, esta deve ser usada
\begin{equation*}
  x_{uk} \leq y_k, \forall u \in V, k \in \{1 ... m\}
\end{equation*}

\item Os Vértices vizinhos não podem ter a mesma cor
\begin{equation*}
  x_{uk} + x_{vk} \leq 1, \forall (u,v) \in E, k \in \{1 ... m\}
\end{equation*}
 
\end{itemize}


\subsubsection*{Função objetivo}
Objetivo: minimizar o número de cores usadas:
\begin{equation}
min\sum_{k=1}^{m}y_k
\end{equation}


\subsection{{[}ss2{]}}
\subsubsection*{Variáveis usadas no modelo}
\subsubsection*{Retrições do modelo}
\subsubsection*{Função objetivo}

\subsection{{[}ss15{]}}
\subsubsection*{Variáveis usadas no modelo}
\subsubsection*{Retrições do modelo}
\subsubsection*{Função objetivo}

\subsection{{[}mn22{]}}
\subsubsection*{Variáveis usadas no modelo}
\subsubsection*{Retrições do modelo}
\subsubsection*{Função objetivo}


% ******************************************************
% Parte 1 resultados
% ******************************************************

\subsection{Resultados}

% Export from lyx and paste here!

\begin{table}[h!]
\begin{centering}
\begin{tabular}{|c|c|c|c|}
\hline 
ID exercício & 1 & 2 & 3\tabularnewline
\hline 
\hline 
{[}nd30{]}  &  &  & \tabularnewline
\hline 
 {[}mn27{]} &  &  & \tabularnewline
\hline 
 {[}ss2{]} &  &  & \tabularnewline
\hline 
{[}ss15{]} &  &  & \tabularnewline
\hline 
 {[}mn22{]} &  &  & \tabularnewline
\hline 
\end{tabular}
\par\end{centering}

\caption{Resultados da parte 1}
\end{table}

% ******************************************************
% Parte 2
% ******************************************************

\subsection{Parte 2}



% ******************************************************
% REFERENCIAS BIBLIOGRÁFICAS
% ******************************************************
% \section{Referências}
\bibliographystyle{plain}
\begin{small}
  \bibliography{referencias}
\end{small}

\end{document}
