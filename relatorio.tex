\documentclass[11pt,letterpaper]{article}
\usepackage[T1]{fontenc}
\usepackage[brazil]{babel}
\usepackage[utf8]{inputenc}

\usepackage{ae,aecompl}
\usepackage{pslatex}
\usepackage{epsfig}
\usepackage{geometry}
\usepackage{url}
\usepackage{textcomp}
\usepackage{ae}
\usepackage{subfig}
\usepackage{indentfirst}
\usepackage{textcomp}
\usepackage{color}
\usepackage{setspace}
\usepackage{verbatim}
\usepackage{mathtools}
\usepackage{amsmath}

% Gráficos
\usepackage{pgfplots}
\pgfplotsset{compat=1.3}
\usepgfplotslibrary{groupplots}

\usepackage{fmtcount} % displaying latex counters

% \usepackage[compact]{titlesec}
% \titlespacing{\section}{0pt}{*0}{*0}
% \titlespacing{\subsection}{0pt}{*0}{*0}
% \titlespacing{\subsubsection}{0pt}{*0}{*0}

\geometry{ 
  letterpaper,	% Formato do papel
  tmargin=25mm,	% Margem superior
  bmargin=5mm,	% Margem inferior
  lmargin=20mm,	% Margem esquerda
  rmargin=20mm,	% Margem direita
  footskip=5mm	% Espaço entre o rodapé e o fim do texto
}
%  ABACO -- Conjunto de macros para desenhar o 'abaco

%  Desenho original de Hans Liesenberg

%  Macros de Tomasz Kowaltowski

%  DCC -- IMECC -- UNICAMP

%  Mar,co de 1988  --  Vers~ao 1.0

% Ajustado para LaTeX da SUN -- Mar,co de 1991

% ---------------------------------------------------------

%  Chamada:   \ABACO{d1}{d2}{d3}{d4}{esc}
%             com:  di's -- os quatro d'igitos;
%	           esc  -- fator de escala

% ---------------------------------------------------------

%  DEFINI,C~OES AUXILIARES

% ---------------------------------------------------------


%  Forma o d'igito pequeno (0 ou 1)

\newcommand{\ABACODP}[1]{%
%
\thicklines
%    
\begin{picture}(8,0)
    \ifcase#1{   %  caso 0
       \put(0,0)    {\line(1,0){4}}
       \multiput(5,0)(2,0){2}{\oval(2,4)}}
    \or{         %  caso 1
       \put(2,0)    {\line(1,0){4}}
       \multiput(1,0)(6,0){2}{\oval(2,4)}}
    \fi
\end{picture}
    } % \ABACODP

% Forma o d'igito grande (0 a 4)

\newcommand{\ABACODG}[1]{%
%
\thicklines
%    
\begin{picture}(14,0)
    \ifcase#1{   % caso 0
       \multiput(1,0)(2,0){5}{\oval(2,4)}}
       \put(10,0)   {\line(1,0){4}}
    \or{         % caso 1
       \multiput(1,0)(2,0){4}{\oval(2,4)}}
       \put(8,0)   {\line(1,0){4}}
       \put(13,0)   {\oval(2,4)}
    \or{         % caso 2
       \multiput(1,0)(2,0){3}{\oval(2,4)}
       \put(6,0)   {\line(1,0){4}}
       \multiput(11,0)(2,0){2}{\oval(2,4)}}
    \or{         % caso 3
       \multiput(1,0)(2,0){2}{\oval(2,4)}
       \put(4,0)   {\line(1,0){4}}
       \multiput(9,0)(2,0){3}{\oval(2,4)}}
    \or{         % caso 4
       \put(1,0)  {\oval(2,4)}}
       \put(2,0)   {\line(1,0){4}}
       \multiput(7,0)(2,0){4}{\oval(2,4)}
    \fi
\end{picture}
    } % \ABACODG
       
% Forma um d'igito (0 a 9)

\newcommand{\ABACOD}[1]{%
%
    \ifnum#1>9
       \errmessage{#1: Argumento invalido para ABACO}
    \fi
    \ifnum#1<0
       \errmessage{#1: Argumento invalido para ABACO}
    \fi
%
\begin{picture}(24,0)
%    
    \ifnum#1<5
       \put(16,0) {\ABACODP{0}}
    \else   
       \put(16,0) {\ABACODP{1}}
    \fi
%    
    \ifnum#1<5
       \put(0,0)  {\ABACODG{#1}}
    \else
       \ifcase#1\or \or \or \or
          \or  \put(0,0)  {\ABACODG{0}}
          \or  \put(0,0)  {\ABACODG{1}}
          \or  \put(0,0)  {\ABACODG{2}}
          \or  \put(0,0)  {\ABACODG{3}}
          \or  \put(0,0)  {\ABACODG{4}}
       \fi
    \fi   
\end{picture}
    } % \ABACOD
    
% -------------------------------------------------

%  DEFINI,C~AO PRINCIPAL
    
\newcommand{\ABACO}[5]{%
    \setlength{\unitlength}{#5mm}
%
    \thinlines
%   
\begin{picture}(28,25)
%   
% moldura
%
% externa
%
        \put(0,0)            {\line(0,1){25}}
        \put(0,0)            {\line(1,0){28}}
        \put(28,0)           {\line(0,1){25}}
        \put(0,25)           {\line(1,0){28}}
% interna
        \put(2,2)            {\line(0,1){21}}
	\put(26,2)           {\line(0,1){21}}
	\put(16,2)           {\line(0,1){21}}
	\put(18,2)           {\line(0,1){21}}
	\put(2,2)            {\line(1,0){14}}
	\put(16,2)           {\line(1,-1){1}}
	\put(17,1)           {\line(1,1){1}}
	\put(18,2)           {\line(1,0){8}}
	\put(2,23)           {\line(1,0){14}}
	\put(16,23)          {\line(1,1){1}}
	\put(17,24)          {\line(1,-1){1}}
	\put(18,23)          {\line(1,0){8}}
	\put(0,0)            {\line(1,1){2}}
	\put(0,25)           {\line(1,-1){2}}
	\put(28,0)           {\line(-1,1){2}}
	\put(28,25)          {\line(-1,-1){2}}
%
%   
% d'igitos
%
%   
       \put(2,20)  {\ABACOD{#1}}
       \put(2,15)  {\ABACOD{#2}}
       \put(2,10)  {\ABACOD{#3}}
       \put(2,5)   {\ABACOD{#4}}
%      
\end{picture}
    } % \ABACO
    
 
\renewcommand{\thetable}{\Roman{table}}
\newcommand{\x} {$\bullet$}


\begin{document}
% CAPA
\thispagestyle{empty}

\begin{minipage}[h]{0.10\linewidth}
  \ABACO{1}{9}{6}{9}{0.5} 
\end{minipage}
\begin{minipage}[h!]{0.7\linewidth}
  \vspace*{\fill}
  \centering
  {\large \textbf{UNIVERSIDADE~ESTADUAL~DE~CAMPINAS}}\\ 
  {\large INSTITUTO~DE~COMPUTAÇÃO}                   
  \vspace*{\fill} 
\end{minipage}
\\\vspace{0.5cm}

\begin{center} 
  \rule{11.0cm}{0.4pt}\vspace*{-\baselineskip}\vspace{-2.0pt}
  \rule{11.0cm}{1.6pt} \vspace*{1.0pt}\\
  {\Large \textsc{Relatório do projeto de MC548}}\\
  \vspace*{-\baselineskip}\vspace{4.2pt} \rule{11.0cm}{0.4pt}
  \vspace*{-\baselineskip}\vspace{3.2pt} \rule{11.0cm}{1.6pt}\\
  {\textsl{}}
  \\\vspace{1cm}
  \begin{tabular}{ll}
    \textbf{Aluno}:   Murilo~Fossa~Vicentini &
    \textbf{RA}:          082335 \\ 
    \textbf{Aluno}:        Tiago~Chedraoui~Silva & 
    \textbf{RA}:        082941 \\
  \end{tabular}
\end{center}
\vspace{0.5cm}

% Sumário
\tableofcontents
\listoftables
\listoffigures 

\vspace{2mm}
\newpage

\section{Integrantes}
\begin{tabular}{rl}
  \textbf{Aluno}:   Murilo~Fossa~Vicentini &
  \textbf{RA}:          082335 \\ 
  \textbf{Aluno}:        Tiago~Chedraoui~Silva & 
  \textbf{RA}:        082941 \\
\end{tabular}


% ******************************************************
% Parte 1
% ******************************************************

\section{Parte 1}

\subsection{{[}nd30{]}}
\subsubsection*{Variáveis usadas no modelo}
\begin{itemize}
\item Para cada aresta $(i,j) \in A$ , criou-se
  a variável binária $y_{ij}$ que assume valor $y_{ij}=1$ se e somente se a aresta (i,j)
  pertence ao caminho mínimo.
\end{itemize}

\subsubsection*{Restrições do modelo}
\begin{itemize}

\item Todo vértice diferente do inicial e do final deve conter ou
  nenhuma aresta entrando e saindo ou uma entrando e saindo.
  \begin{equation*}
    \sum_{i \in V}^{m}y_{ik}=\sum_{j \in V}^{m}y_{kj},\quad \forall k \in V,
    \forall (i,k) \; \text{e} \; (k,j) \in A 
  \end{equation*}

\item Peso total do caminho não deve exceder K
  \begin{equation*}
    \sum_{{i,j} \in A}w_{i,j}y_{i,j} \leq K
  \end{equation*}

\item Deve existir uma aresta que sai de s
  \begin{equation*}
    \sum_{j \in V}y_{s,j} = 1 
  \end{equation*}

\item Deve existir uma aresta que chega em t
  \begin{equation*}
    \sum_{j \in V}y_{j,t} = 1 
  \end{equation*}

\end{itemize}


\subsubsection*{Função objetivo}
Objetivo: minimizar o custo do caminho 
\begin{equation}
  min\sum_{{i,j} \in A}c_{i,j}y_{i,j}
\end{equation}

\subsection{{[}mn27{]}}

\subsubsection*{Variáveis usadas no modelo}
\begin{itemize}
\item Para cada vértice $u \in V$ e para cada cor $k \in \{1,2,...,m\}$, criou-se
  a variável binária $x_{uk}$ que assume valor $x_{uk}=1$ se e somente se o vértice $u$ foi
  colorido com a cor $k$.

\item Criou-se uma variável binária $y_k$ para toda cor $k \in
  \{1,2,...,m\}$. $y_k=1$ se e somente se pelo menos um vértice recebeu essa cor.
\end{itemize}

\subsubsection*{Restrições do modelo}
\begin{itemize}

\item Todo vértice deve receber exatamente uma cor
  \begin{equation*}
    \sum_{k=1}^{m}x_{uk}=1,\quad \forall u \in V
  \end{equation*}

\item Se um vértice recebe a cor k, esta deve ser usada
  \begin{equation*}
    x_{uk} \leq y_k, \quad\forall u \in V, k \in \{1 ... m\}
  \end{equation*}

\item Os Vértices vizinhos não podem ter a mesma cor
  \begin{equation*}
    x_{uk} + x_{vk} \leq 1, \quad\forall (u,v) \in E, k \in \{1 ... m\}
  \end{equation*}
  
\end{itemize}


\subsubsection*{Função objetivo}
Objetivo: minimizar o número de cores usadas:
\begin{equation}
  min\sum_{k=1}^{m}y_k
\end{equation}


\subsection{{[}ss2{]}}
\subsubsection*{Variáveis usadas no modelo}
\begin{itemize}
\item Criou-se uma variável binária $x_{ij}$ para toda tarefa $i, j
  \in T$ que recebe valor $x_{ij}=1$ se e somente se a
  tarefa $i$ precede $j$.

\item Para cada tarefa $i \in T$ criou-se uma variável binária $y_{i}$
  que recebe valor $y_{i}=1$ se e somente se a tarefa i não cumpriu o deadline.
\end{itemize}

\subsubsection*{Restrições do modelo}
\begin{itemize}

\item Todo par de tarefas $(i,j)$ deve ter uma precedência, em que se $i$
  precede $j$, $j$ não pode preceder $i$. 
  \begin{equation*}
    x_{ji}+x_{ij}=1, \quad\forall i,j \in T
  \end{equation*}

\item Todo par de tarefas $(i,j)$ deve ter uma transitoriedade de precedência, em que se $i$
  precede $j$, e $j$ precede $k$, $i$ precede $k$. Se $x_{ij}=1$ e
  $x_{jk}=1$, então $x_{ik}=1$.
  \begin{equation*}
    x_{ij}+x_{jk}-1 \leq x_{i,k}, \quad \forall i,j,k \in T
  \end{equation*}


\item Para cada par de tarefas $(i,j)$ em S, a tarefa $i$,
  obrigatoriamente tem que preceder $j$. 
  \begin{equation*}
    x_{ij} = 1,\quad \forall (i,j) \in S
  \end{equation*}

\item Se uma $j$ tarefa é precedida por outras n tarefas, o tempo de
  término da tarefa $j$ deve ser no mínimo o tempo de execução de todas
  as tarefas predecessoras, mais o seu tempo para ser executada. Se
  for esse término for menor que o deadline, $y_{j}=0$, senão $y_{j}=1$ 
  \begin{equation*}
    \sum_{i \in T, i \neq j} x_{ij}*t_{i} \leq d_{j}-t_{j}+ M*y_{j}\quad \forall j \in T
  \end{equation*}
  
  Em que M é um número grande e para calculá-lo somou-se todos os tempos
  das tarefas mais um. 
  \begin{equation*}
    M = \sum_{i \in T} (t_{i}) +1
  \end{equation*}
  
\end{itemize}

\subsubsection*{Função objetivo}
Objetivo: minimizar o número de tarefas que terminem fora do prazo:
\begin{equation}
  min\sum_{i=1}^{n}y_i
\end{equation}

\subsection{{[}ss15{]}}
\subsubsection*{Variáveis usadas no modelo}
\begin{itemize}
\item Criou-se uma variável binária $x_{i,j,k}$ para toda tarefa $k \in T$ de todo projeto $i, j  \in J$ que recebe valor $x_{i,j,k}=1$ se e somente se a tarefa $k$ do projeto $i$ precede a tarefa $k$ do projeto $j$.

\item Para cada projeto $j \in J$ e cada tarefa $i \in T$ criou-se uma variável inteira $begin_{j,i}$
  que recebe valor o valor de início da tarefa $i$ do projeto $j$.

\item Criou-se uma variável inteira $fim$ que recebe o tempo de término
  do último projeto.

\end{itemize}

\subsubsection*{Restrições do modelo}
\begin{itemize}
\item Toda tarefa dos pares de projetos $(i,j)$ deve ter uma precedência, em que se $i$
  precede $j$, $j$ não pode preceder $i$. 
  \begin{equation*}
    x_{j,i,k}+x_{i,j,k}=1, \quad \forall i,j \in J
  \end{equation*}

\item Toda tarefa $i \in T$ do projeto  $j \in J$ tem um tempo mínimo de início que é equivalente ao tempo de início da tarefa que a antecede $i -1$ mais o tempo da execução da tarefa predecessora $ t_{j,i-1}$ para o mesmo projeto. 
  \begin{equation*}
    begin_{j,i} \geq begin_{j,i-1}+ t_{j,i-1}, \quad\forall i \in T, \forall j \in J
  \end{equation*}

\item Se um projeto $j$ precede um projeto $i$ o tempo de término da
  tarefa $k$ do projeto $i$ deve ser maior que o tempo de término da
  tarefa $z$ do projeto $j$ mais o seu tempo de execução.
  
  \begin{equation*}
    begin_{j,i} + t_{j,i} \leq begin_{k,i}+(1-x_{j,k,i})*M,\quad \forall i \in T, \forall j,k \in J  : k!=j 
  \end{equation*}


  Em que M é um número grande e para calculá-lo somou-se todos os tempos
  das tarefas de todos os projetos. 

\item A tarefa $k$ do projeto $j$ precede a tarefa $z$ do mesmo
  projeto, logo o tempo de término da
  tarefa $z$ deve ser maior que o tempo de término da
  tarefa $k$ mais o seu tempo de execução.
  
  \begin{equation*}
    begin_{k,i} + t_{k,i} \leq begin_{j,i}+x_{j,k,i}*M,\quad \forall i \in T , \forall j,k \in J : k!=j 
  \end{equation*}

  Em que M é um número grande e para calculá-lo somou-se todos os tempos
  das tarefas de todos os projetos. 


  \begin{equation*}
    M = \sum_{j \in J,i \in T} (t_{j,i}) 
  \end{equation*}

\item O tempo total da execução dos projetos deve ser igual ao tempo
  do último terminar.
  
  \begin{equation*}
    fim \geq begin_{j,m}+ t_{j,m},\quad \forall j \in J
  \end{equation*}
  Em que $m$ é o tempo em que a última tarefa do projeto é executada
  (tarefa no último processador).
\end{itemize}


\subsubsection*{Função objetivo}
Objetivo: minimizar o tempo de término de todos os projetos:
\begin{equation}
  \min fim
\end{equation}


\subsection{{[}mn22{]}}
\subsubsection*{Variáveis usadas no modelo}
\begin{itemize}
\item Para cada máquina $m \in V$ e para cada sala $r \in
  \{1,2,...,|V|\}$, criou-se a variável binária $x_{mr}$ que assume valor
  $x_{mr}=1$ se e somente se  a máquina $m$
  foi colocada na sala $r$.


\item Para cada peça $p \in U$ e para cada sala $r \in
  \{1,2,...,|U|\}$, criou-se a variável binária $y_{pr}$ que assume valor
  $y_{pr}=1$ se e somente se a peça $p$  foi colocada na sala $r$.

\item Criou-se uma variável binária $rdiff_{mp}$ para
  cada aresta da máquina $(m,p) \in  E$ para a qual
  $rdiff_{mp}=1$ se e somente se a máquina e a peça especificada pela aresta estão em salas diferentes.

\end{itemize}

\subsubsection*{Restrições do modelo}
\begin{itemize}
\item Toda máquina deve estar em uma única sala
  \begin{equation*}
    \sum_{r \in  \{1,2,...,|V|\}}x_{mr}=1,\quad \forall m \in V
  \end{equation*}

\item Toda peça deve estar em uma única sala
  \begin{equation*}
    \sum_{r \in  \{1,2,...,|V|\}}y_{pr}=1, \quad\forall p \in U
  \end{equation*}

\item Número de máquinas por sala não deve exceder limite K
  \begin{equation*}
    \sum_{m \in V }x_{mr}\leq K,\quad \forall r \in \{1,2,...,|V|\}
  \end{equation*}

\item Se uma máquina estiver em sala diferente de sua peça, $rdiff=1$
  \begin{equation*}
    rdiff_{mp}\geq x_{mr}-y_{pr},\quad \forall r \in \{1,2,...,|V|\}, \forall (p,m) \in E
  \end{equation*}

\end{itemize}

\subsubsection*{Função objetivo}
Objetivo: minimizar a soma do custo de transporte de uma peça $p$ para
a mesma sala da máquina $m$:
\begin{equation}
  min\sum_{(i,j) \in E}c_{i,j}*rdiff_{i,j}
\end{equation}


% ******************************************************
% Parte 1 resultados
% ******************************************************

\subsection{Resultados}


% reset the counter to the first footnote's value
\addtocounter{footnote}{1}

\begin{table}[h!]
  \begin{centering}
    \begin{tabular}{|c|c|c|c|}
      \hline 
      ID exercício & 1 & 2 & 3\tabularnewline
      \hline 
      \hline 
      {[}nd30{]}  & 3 & 13  & 21 \tabularnewline
      \hline 
      {[}mn27{]} & 3 & 7 & 13$^{\decimal{footnote}}$\addtocounter{footnote}{1} \tabularnewline
      \hline 
      {[}ss2{]} & 1 & 6  & 17 \tabularnewline
      \hline 
      {[}ss15{]} & 8 & 165  & 245$^{\decimal{footnote}}$\addtocounter{footnote}{1}  \tabularnewline
      \hline 
      {[}mn22{]} & 1 & 4131  & 3691$^{\decimal{footnote}}$\addtocounter{footnote}{1} \tabularnewline
      \hline 
    \end{tabular}
    \par\end{centering}
  \caption{Resultados da parte 1 - Modelagem gmpl}
\end{table}

\begin{table}[h!]
  \begin{centering}
    \begin{tabular}{|c|c|c|c|}
      \hline 
      ID exercício & 1 & 2 & 3\tabularnewline
      \hline 
      \hline 
      {[}nd30{]}  & 3 & 13  & 21 \tabularnewline
      \hline 
      {[}mn27{]} & 3 & 7 & Não terminou \tabularnewline
      \hline 
      {[}ss2{]} & 1 & Não terminou  & Não terminou \tabularnewline
      \hline 
      {[}ss15{]} & 8 & Não terminou  & Não terminou  \tabularnewline
      \hline 
      {[}mn22{]} & 1 & Não terminou  & Não terminou \tabularnewline
      \hline 
    \end{tabular}
    \par\end{centering}
  \caption{Resultados da parte 1 - Modelagem planilha}
\end{table}

% \newpage
\addtocounter{footnote}{-3} 
\footnotetext{Rodou-se durante 30 minutos para somente 13 cores (13
  cores havia sido uma solução encontrada anteriormente, diminui-se os
  número de cores para o número de restrições diminuir e assim o
  processamento ser mais rápido);}
\addtocounter{footnote}{1} 
\footnotetext{Rodou-se durante 30 minutos para somente 4 salas;}
\addtocounter{footnote}{1} 
\footnotetext{ Utilizando a abordagem apresentada na seção do
  problema, e utilizando-a durante 8 horas em um servidor do
  laboratório de redes da Unicamp encontrou-se o resultado não ótimo
  257. Contudo, se considerarmos somente a ordem dos projetos a solução é
  encontrada em questão de segundos e o resultado é melhor, 245, porém
  talvez não seja o ótimo. Como
  sabe-se que esse resultado é um resultado para a abordagem 
  da seção do problema (seção 2.4) esse resultado foi considerado o melhor que encontramos;}
% ******************************************************
% Parte 2
% ******************************************************
\section{Parte 2}

Na segunda parte desenvolveu-se uma heurística gulosa, no qual
ordenava-se os shards em relação ao ganho do shard com o mínimo de
custo entre o satélite vertical e o horizontal.
A partir disso pegava-se os shards até que não fossem mais possível.
Posteriormente, desenvolveu-se uma busca local a partir da solução
gulosa.

Na busca local retirava-se um shard da solução ótima e tentava colocar
outros candidatos no lugar. A remoção de um shard da solução ótima é realizada do que possuir
menor relação custo/benefício para a melhor, ou seja, se um shard
ocupa mais espaço e possibilita um menor ganho.

A lista de candidatos (LRC) foi limitada a 10 candidatos, estes
constituídos dos 10 que possuem melhores relação custo/benefício.
A escolha dos candidatos foi novamente feito através de uma busca
gulosa. Percorre-se a lista lrc e tenta inserir cada um.

Isso era feito para todos os shards da solução ótima, se o resultado
fosse melhorado ele era salvo. Caso não fosse, o estado anterior era
restaurado, de modo que a mudança na solução que foi ruim, fosse
descartado. Feito isso uma vez, isso é repetido até
que o tempo limite seja atingido.

A idéia inicial do algoritmo guloso era dar uma solução boa em tempo
curto e a busca local seria justamente inserir uma variabilidade no algoritmo.

\subsection{Estruturas implementadas}
Foram utilizadas algumas estruturas, dentre elas:
\begin{description}
\item[Estrutura Shard] contendo dois inteiros x e y correspondentes aos
satélites que conseguem tirar foto do shard. Dois inteiro hcost e
vcost representam o custo de armazenamento dos shards para cada
satélite x e y. E uma flag active, se o shard já está ou não na
solução.
\item[Estrutura Satélite] Contem dois inteiro que contem o tamanho da
  memória de cada satélite sendo um na horizontal e outro na vertical.
\item[Estrutura CSShard] Contém shards inseridos na solução, contém x
  e y do shard, a direção do satélite responsável pela foto, o indíce
  idx do shard na estrutura com todos os shards.
\item[Estrutura Solution] Contém valor da solução ótima encontrada, o
  número de shards inseridos na solução, e um vetor de CSshards.

\end{description}


\subsection{Análise de complexidade do algoritmo}
Denotaremos a $|Sat|= m $ e $|Shards|= n $. 

\begin{enumerate}
\item função main $O(n^2+nm)$
  \begin{enumerate}
  \item função get\underline{ }args: $O(1)$
  \item função read\underline{ }instances: $O(n+m)$
  \item função quicksort: $O(n^2)$ 
  \item função Greedy\underline{ }solver: $O(n)$
  \item  Local\underline{ }search: $O(n^2+nm)$
    \begin{enumerate}
    \item copy\underline{ }sol: $O(n)$
    \item save\underline{ }state: $O(n+m)$
    \item recover\underline{ }statel: $O(n+m)$
    \item find\underline{ }lrc: $O(n)$
    \item choose\underline{ }lrc: $O(n)$
    \item save\underline{ }sol: $O(n)$ 
    \end{enumerate}
  \end{enumerate}
\end{enumerate}

Observações:
\begin{itemize}
\item Local\underline{ }search executa $|n|$ as  funções de $i$ até $vi$
  listadas acima;

\item A função main tem sua complexidade delimitada pela função
  Local\underline{ }search; e exata é executada até
  que o tempo seja atingido.
\end{itemize}

\subsection{Resultados}

Apresentamos dois resultados, o primeiro para a solução gulosa (ver
tabela \ref{tab:gul}) e o segundo
aplicando a busca local (ver
tabela \ref{tab:loc})sobre a solução gulosa encontrada.

Posteriormente, utilizando duas instâncias de entrada (big-0 e
small-0) projetou-se um gráfico (ver figuras  \ref{graf:big} e \ref{graf:small}) com a melhora da
solução em relação ao número de iterações para a heurística.

% reset the counter to the first footnote's value
\addtocounter{footnote}{1}

\begin{table}[h!]
  \begin{centering}
    \begin{tabular}{|c|c|c|}
      \hline 
      Instância & Melhor solução encontrada & Parâmetro Tempo  \tabularnewline
      \hline 
      \hline 
      small-0 & 8874 & 5s \tabularnewline %8882
      \hline 
      small-1 & 56435 & 5s  \tabularnewline %56435
      \hline 
      small-2 & 29430 & 5s \tabularnewline %29430
      \hline 
      small-3 & 14848 & 5s \tabularnewline %14848
      \hline 
      small-4 & 31403 & 5s \tabularnewline %31403
      \hline 
      medium-0 & 3354366 & 5s \tabularnewline %3354366
      \hline 
      medium-1 & 1334090 & 5s \tabularnewline %1334090
      \hline 
      medium-2 & 3590494 & 5s \tabularnewline %3590494
      \hline 
      medium-3 & 1927658 & 5s \tabularnewline %1927658
      \hline 
      medium-4 & 3696487 & 5s \tabularnewline %3696487
      \hline 
      medium-5 & 2350322 & 5s \tabularnewline %2350322
      \hline 
      medium-6 & 743412 & 5s \tabularnewline %743412
      \hline 
      medium-7 & 1157211 & 5s \tabularnewline %1157211
      \hline 
      big-0 & 10148418 & 5s \tabularnewline %10148418
      \hline 
      big-1 & 19194050 & 5s \tabularnewline %19194050
      \hline 

    \end{tabular}
    \par\end{centering}
  \caption{Resultados da parte 2 - Somente parte gulosa}
  \label{tab:gul}
\end{table}


\begin{table}[h!]
  \begin{centering}
    \begin{tabular}{|c|c|c|}
      \hline 
      Instância & Melhor solução encontrada & Parâmetro Tempo  \tabularnewline
      \hline 
      \hline 
      small-0 & 9937 & 2 min  \tabularnewline %max 13666
      \hline 
      small-1 & 58763 & 2 min  \tabularnewline %max 76022
      \hline 
      small-2 & 33013 & 2 min  \tabularnewline %max 39212
      \hline 
      small-3 & 17449 & 2 min  \tabularnewline %max 21307
      \hline 
      small-4 & 33029 & 2 min  \tabularnewline %max 49391
      \hline 
      medium-0 &  3484321 & 2 min  \tabularnewline %max 4718409
      \hline 
      medium-1 & 1394863 & 2 min  \tabularnewline %max 1845683
      \hline 
      medium-2 & 3756501 & 2 min  \tabularnewline %max 5002484
      \hline 
      medium-3 & 2001891 & 2 min  \tabularnewline %max 2804919
      \hline 
      medium-4 & 3858310 & 2 min  \tabularnewline %max  5021152
      \hline 
      medium-5 & 2454255 & 2 min  \tabularnewline %max 3194956
      \hline 
      medium-6 & 779016 & 2 min  \tabularnewline %max 1139797
      \hline 
      medium-7 & 1203455 & 2 min  \tabularnewline %max 1690257
      \hline 
      big-0 & 10538294 & 2 min  \tabularnewline %max 14510868
      \hline 
      big-1 & 19714711 & 2 min  \tabularnewline %max 25453577 
      \hline 

    \end{tabular}
    \par\end{centering}
  \caption{Resultados da parte 2 - Parte gulosa e busca local}
  \label{tab:loc}
\end{table}


\begin{figure}[h!]
  \begin{center}
    \begin{tikzpicture}
      \begin{axis}[
	x tick label style={
          /pgf/number format/1000 sep=},
	ylabel= Valor da função objetivo,
	enlargelimits=0.15,
        height=8cm,
        width=16cm,
	bar width=7pt,
        legend pos=south east,
        xlabel=iterações
        % xtick ={131, 133, 134, 135, 138, 139,142,144,145,275,277,278,391},
        % xticklabels ={131, 133, 134, 135, 138,139,142,144,145,275,277,278,391},
        ]

        \addplot  table[x=it,y=ganho] {big.dat};
        \legend{Big-0}
      \end{axis}
    \end{tikzpicture}
  \end{center}
  \caption{Melhora com busca local - Ganho fotografando shards -
    Instância big-0}
  \label{graf:big}
\end{figure}


\begin{figure}[h!]
  \begin{center}
    \begin{tikzpicture}
      \begin{axis}[
	x tick label style={
          /pgf/number format/1000 sep=},
	ylabel= Valor da função objetivo,
	enlargelimits=0.15,
        height=8cm,
        width=16cm,
	bar width=7pt,
        legend pos=south east,
        xlabel=iterações
        % xtick ={131, 133, 134, 135, 138, 139,142,144,145,275,277,278,391},
        % xticklabels ={131, 133, 134, 135, 138,139,142,144,145,275,277,278,391},
        ]

        \addplot  table[x=it,y=ganho] {small.dat};
        \legend{Small-0}
      \end{axis}
    \end{tikzpicture}
  \end{center}
  \caption{Melhora com busca local - Ganho fotografando shards - Instância small-0}
  \label{graf:small}
\end{figure}

\section{Ambientes computacionais}
\begin{enumerate}
\item Máquina 1

  \begin{description}
  \item[Memória]2.0 GB
  \item[Processador]  Intel® Core™2 Duo CPU T5250 @ 1.50GHz × 2
  \item[Gráfico] Intel® 965GM   
  \item[Tipo SO] 32 bits   
  \item[SO] Fedora 15 (Lovelock)   
  \item[Compilador] gcc (GCC) 4.6.0 20110530 (Red Hat 4.6.0-9)
  \item[Linguagem de programação] C
  \end{description}

\end{enumerate}

% ******************************************************
% REFERENCIAS BIBLIOGRÁFICAS
% ******************************************************
% \section{Referências}
% \bibliographystyle{plain}
% \begin{small}
%   \bibliography{referencias}
% \end{small}

\end{document}
